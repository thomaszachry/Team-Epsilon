In this section, you will describe all of the various artifacts that you will generate and maintain during the project lifecycle. Describe the purpose of each item below, how the content will be generated, where it will be stored, how often it will be updated, etc. 

\subsection{Project Charter}
The charter gives the terms of reference, being able to outline roles and responsibilities, objectives, and identify stakeholders if any. It will be generated as a team effort. It has no need to be maintained as it is a one-time document, beyond revising drafts before it is due. It will be stored in GitHub team repository.

\subsection{Product Backlog}
The backlog is a prioritized features list, generated each sprint cycle with team input. It will be maintained on a daily, weekly, and cyclic basis. Stored in Leankit project tracking app.

\subsection{Sprint Planning}
Sprint planning is used to plan our tactics for each sprint and how to do so and what amount of time we should dedicate to doing so. It will be generated as a team effort at the beginning of each sprint and maintained (and stored) daily on the Leankit tracking app.

\subsubsection{Sprint Goal}
The sprint goal is out goal to achieve after each sprint time period. It will be generated at the beginning of each sprint cycle through team input. Progress will be maintained daily with workup to final production of the goal. Whether the goal has been achieved is agreed upon by each team member.

\subsubsection{Sprint Backlog}
The spring backlog is a list of tasks to be completed by the end of each spring. It will be generated as a team effort at the beginning of each sprint and maintained daily on the Leankit tracking app.

\subsubsection{Task Breakdown}
The task breakdown is a deliverable-oriented hierarchical decomposition of the work to be executed by the project team to accomplish the project objectives and create the required deliverables. It will be delegated and agreed upon at the beginning of sprint cycles and team meetings. Tracked through Leankit and stored on the online repository.

\subsection{Sprint Burndown Charts}
The chart displays the remaining size of all stories in a sprint backlog that needs to be done. It will be generated at end of each sprint cycle, maintained after each and every cycle, and stored on the repository.

\begin{figure}[h!]
    \centering
    \includegraphics[width=0.5\textwidth]{images/test_image}
    \caption{Example sprint burndown chart}
\end{figure}

\subsection{Sprint Retrospective}
The sprint retrospective allows the team to review their progress during a spring and improve upon or learn from it during the next sprint. It will be generated at end of each sprint cycle, maintained after each and every cycle, and stored on the repository.

\subsection{Individual Status Reports}
These status reports allow individual team members to be reviewed and recieve input on how to improve their contribution. Done at each and every team meeting and stored in the Github repository.

\subsection{Engineering Notebooks}
Engineering notebooks allow professional documentation of the project development. Maintained at every thought by each member of the team at their digression.

\subsection{Closeout Materials}
Will be mechanical designs, code, mockups, protoypes, any documentations used to make the product, the and user manual. Generated as they are done at the time. 

\subsubsection{System Prototype}
It will be the mockup before the actual building. Stored in Senior design lab.

\subsubsection{Project Poster}
The poster will aid in presenting the project and its features. It will be generated by the team stored in the senior design lab.

\subsubsection{Web Page}
The web page will aid in presenting the project and its features. It will be generated by and for the team.

\subsubsection{Demo Video}
The demo video will aid in presenting the project and its features. It will be stored on team webpage made by and for the team.

\subsubsection{Source Code}
The source code is needed to program the project to perform. All code including test code stored and maintained on team repository.

\subsubsection{Source Code Documentation}
The documentation will aid in context and readibility of the source code and record any necessary information as needed. Integrated in code and supporting documentation as well as maintained and updated as needed on the repository.

\subsubsection{CAD files}
These files can be used to analyze and mockup the project for simulation. If generated by team and used for testing purposes, will be stored on team repository and made available on team webpage.

\subsubsection{User Manual}
The user manual will aid a person in being aboe to use the final prouct. Created by, for, and about us made available on the webpage and stored on the repository.