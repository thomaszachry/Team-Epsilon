Discuss the state-of-the-art with respect to your product. What solutions currently exist, and in what form (academic research, enthusiast prototype, commercially available, etc)? Include references and citations as necessary \cite{Rubin2012}.
\\
\\
Cartesian and six-axis robots, along with SCARAs, automate tasks to accelerate cycle times, increase throughput, and eliminate bottlenecks.
\\
\\
Cartesian robots, sometimes called gantry robots, are mechatronic devices that use motors and linear actuators to position a tool. They make linear movements in three axes, X, Y, and Z. Physical scaffolding forms a framework that anchors and supports the axes and payload. Certain applications, such as machining tightly toleranced parts, require full support of the base axis, usually the X axis. In contrast, other applications, such as picking bottles off a conveyor, require less precision, so the framework only needs to support the base axis in compliance with the actuator's manufacturer recommendations. Cartesian-robot movements stay within the framework's confines, but the framework can be mounted horizontally or vertically, or even overhead in certain gantry configurations.
\\
\\
Six-axis robots move forward and back, up and down, and can yaw, pitch, and roll to offer more directional control than SCARAs. This is suitable for complex movements that simulate a human arm - reaching under something to grab a part and place it on a conveyor, for example. The additional range of movement also lets six-axis robots service a larger volume than SCARAs can. Six-axis robots often execute welding, palletizing, and machine tending. Programming their movements in 3D is complex, so software typically maps the motion to a set of world coordinates in which the origin sits on the pedestal's first joint axis
\cite{Vaughn2013}.